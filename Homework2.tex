% Options for packages loaded elsewhere
\PassOptionsToPackage{unicode}{hyperref}
\PassOptionsToPackage{hyphens}{url}
%
\documentclass[
]{article}
\usepackage{amsmath,amssymb}
\usepackage{lmodern}
\usepackage{iftex}
\ifPDFTeX
  \usepackage[T1]{fontenc}
  \usepackage[utf8]{inputenc}
  \usepackage{textcomp} % provide euro and other symbols
\else % if luatex or xetex
  \usepackage{unicode-math}
  \defaultfontfeatures{Scale=MatchLowercase}
  \defaultfontfeatures[\rmfamily]{Ligatures=TeX,Scale=1}
\fi
% Use upquote if available, for straight quotes in verbatim environments
\IfFileExists{upquote.sty}{\usepackage{upquote}}{}
\IfFileExists{microtype.sty}{% use microtype if available
  \usepackage[]{microtype}
  \UseMicrotypeSet[protrusion]{basicmath} % disable protrusion for tt fonts
}{}
\makeatletter
\@ifundefined{KOMAClassName}{% if non-KOMA class
  \IfFileExists{parskip.sty}{%
    \usepackage{parskip}
  }{% else
    \setlength{\parindent}{0pt}
    \setlength{\parskip}{6pt plus 2pt minus 1pt}}
}{% if KOMA class
  \KOMAoptions{parskip=half}}
\makeatother
\usepackage{xcolor}
\IfFileExists{xurl.sty}{\usepackage{xurl}}{} % add URL line breaks if available
\IfFileExists{bookmark.sty}{\usepackage{bookmark}}{\usepackage{hyperref}}
\hypersetup{
  pdftitle={PSTAT 231 Homework 2},
  pdfauthor={Dylan Berneman},
  hidelinks,
  pdfcreator={LaTeX via pandoc}}
\urlstyle{same} % disable monospaced font for URLs
\usepackage[margin=1in]{geometry}
\usepackage{color}
\usepackage{fancyvrb}
\newcommand{\VerbBar}{|}
\newcommand{\VERB}{\Verb[commandchars=\\\{\}]}
\DefineVerbatimEnvironment{Highlighting}{Verbatim}{commandchars=\\\{\}}
% Add ',fontsize=\small' for more characters per line
\usepackage{framed}
\definecolor{shadecolor}{RGB}{248,248,248}
\newenvironment{Shaded}{\begin{snugshade}}{\end{snugshade}}
\newcommand{\AlertTok}[1]{\textcolor[rgb]{0.94,0.16,0.16}{#1}}
\newcommand{\AnnotationTok}[1]{\textcolor[rgb]{0.56,0.35,0.01}{\textbf{\textit{#1}}}}
\newcommand{\AttributeTok}[1]{\textcolor[rgb]{0.77,0.63,0.00}{#1}}
\newcommand{\BaseNTok}[1]{\textcolor[rgb]{0.00,0.00,0.81}{#1}}
\newcommand{\BuiltInTok}[1]{#1}
\newcommand{\CharTok}[1]{\textcolor[rgb]{0.31,0.60,0.02}{#1}}
\newcommand{\CommentTok}[1]{\textcolor[rgb]{0.56,0.35,0.01}{\textit{#1}}}
\newcommand{\CommentVarTok}[1]{\textcolor[rgb]{0.56,0.35,0.01}{\textbf{\textit{#1}}}}
\newcommand{\ConstantTok}[1]{\textcolor[rgb]{0.00,0.00,0.00}{#1}}
\newcommand{\ControlFlowTok}[1]{\textcolor[rgb]{0.13,0.29,0.53}{\textbf{#1}}}
\newcommand{\DataTypeTok}[1]{\textcolor[rgb]{0.13,0.29,0.53}{#1}}
\newcommand{\DecValTok}[1]{\textcolor[rgb]{0.00,0.00,0.81}{#1}}
\newcommand{\DocumentationTok}[1]{\textcolor[rgb]{0.56,0.35,0.01}{\textbf{\textit{#1}}}}
\newcommand{\ErrorTok}[1]{\textcolor[rgb]{0.64,0.00,0.00}{\textbf{#1}}}
\newcommand{\ExtensionTok}[1]{#1}
\newcommand{\FloatTok}[1]{\textcolor[rgb]{0.00,0.00,0.81}{#1}}
\newcommand{\FunctionTok}[1]{\textcolor[rgb]{0.00,0.00,0.00}{#1}}
\newcommand{\ImportTok}[1]{#1}
\newcommand{\InformationTok}[1]{\textcolor[rgb]{0.56,0.35,0.01}{\textbf{\textit{#1}}}}
\newcommand{\KeywordTok}[1]{\textcolor[rgb]{0.13,0.29,0.53}{\textbf{#1}}}
\newcommand{\NormalTok}[1]{#1}
\newcommand{\OperatorTok}[1]{\textcolor[rgb]{0.81,0.36,0.00}{\textbf{#1}}}
\newcommand{\OtherTok}[1]{\textcolor[rgb]{0.56,0.35,0.01}{#1}}
\newcommand{\PreprocessorTok}[1]{\textcolor[rgb]{0.56,0.35,0.01}{\textit{#1}}}
\newcommand{\RegionMarkerTok}[1]{#1}
\newcommand{\SpecialCharTok}[1]{\textcolor[rgb]{0.00,0.00,0.00}{#1}}
\newcommand{\SpecialStringTok}[1]{\textcolor[rgb]{0.31,0.60,0.02}{#1}}
\newcommand{\StringTok}[1]{\textcolor[rgb]{0.31,0.60,0.02}{#1}}
\newcommand{\VariableTok}[1]{\textcolor[rgb]{0.00,0.00,0.00}{#1}}
\newcommand{\VerbatimStringTok}[1]{\textcolor[rgb]{0.31,0.60,0.02}{#1}}
\newcommand{\WarningTok}[1]{\textcolor[rgb]{0.56,0.35,0.01}{\textbf{\textit{#1}}}}
\usepackage{graphicx}
\makeatletter
\def\maxwidth{\ifdim\Gin@nat@width>\linewidth\linewidth\else\Gin@nat@width\fi}
\def\maxheight{\ifdim\Gin@nat@height>\textheight\textheight\else\Gin@nat@height\fi}
\makeatother
% Scale images if necessary, so that they will not overflow the page
% margins by default, and it is still possible to overwrite the defaults
% using explicit options in \includegraphics[width, height, ...]{}
\setkeys{Gin}{width=\maxwidth,height=\maxheight,keepaspectratio}
% Set default figure placement to htbp
\makeatletter
\def\fps@figure{htbp}
\makeatother
\setlength{\emergencystretch}{3em} % prevent overfull lines
\providecommand{\tightlist}{%
  \setlength{\itemsep}{0pt}\setlength{\parskip}{0pt}}
\setcounter{secnumdepth}{-\maxdimen} % remove section numbering
\ifLuaTeX
  \usepackage{selnolig}  % disable illegal ligatures
\fi

\title{PSTAT 231 Homework 2}
\author{Dylan Berneman}
\date{}

\begin{document}
\maketitle

{
\setcounter{tocdepth}{2}
\tableofcontents
}
\hypertarget{linear-regression}{%
\subsection{Linear Regression}\label{linear-regression}}

For this lab, we will be working with a data set from the UCI
(University of California, Irvine) Machine Learning repository
(\href{http://archive.ics.uci.edu/ml/datasets/Abalone}{see website
here}). The full data set consists of \(4,177\) observations of abalone
in Tasmania. (Fun fact:
\href{https://en.wikipedia.org/wiki/Tasmania}{Tasmania} supplies about
\(25\%\) of the yearly world abalone harvest.)

\begin{figure}
\centering
\includegraphics[width=1.58333in,height=\textheight]{https://cdn.shopify.com/s/files/1/1198/8002/products/1d89434927bffb6fd1786c19c2d921fb_2000x_652a2391-5a0a-4f10-966c-f759dc08635c_1024x1024.jpg?v=1582320404}
\caption{\emph{Fig 1. Inside of an abalone shell.}}
\end{figure}

The age of an abalone is typically determined by cutting the shell open
and counting the number of rings with a microscope. The purpose of this
data set is to determine whether abalone age (\textbf{number of rings +
1.5}) can be accurately predicted using other, easier-to-obtain
information about the abalone.

The full abalone data set is located in the
\texttt{\textbackslash{}data} subdirectory. Read it into \emph{R} using
\texttt{read\_csv()}. Take a moment to read through the codebook
(\texttt{abalone\_codebook.txt}) and familiarize yourself with the
variable definitions.

Make sure you load the \texttt{tidyverse} and \texttt{tidymodels}!

\hypertarget{question-1}{%
\subsubsection{Question 1}\label{question-1}}

Your goal is to predict abalone age, which is calculated as the number
of rings plus 1.5. Notice there currently is no \texttt{age} variable in
the data set. Add \texttt{age} to the data set.

Assess and describe the distribution of \texttt{age}.

\begin{Shaded}
\begin{Highlighting}[]
\FunctionTok{head}\NormalTok{(abalone)}
\end{Highlighting}
\end{Shaded}

\begin{verbatim}
## # A tibble: 6 x 9
##   type  longest_shell diameter height whole_weight shucked_weight viscera_weight
##   <chr>         <dbl>    <dbl>  <dbl>        <dbl>          <dbl>          <dbl>
## 1 M             0.455    0.365  0.095        0.514         0.224          0.101 
## 2 M             0.35     0.265  0.09         0.226         0.0995         0.0485
## 3 F             0.53     0.42   0.135        0.677         0.256          0.142 
## 4 M             0.44     0.365  0.125        0.516         0.216          0.114 
## 5 I             0.33     0.255  0.08         0.205         0.0895         0.0395
## 6 I             0.425    0.3    0.095        0.352         0.141          0.0775
## # ... with 2 more variables: shell_weight <dbl>, rings <dbl>
\end{verbatim}

\begin{Shaded}
\begin{Highlighting}[]
\NormalTok{abalone[}\StringTok{\textquotesingle{}age\textquotesingle{}}\NormalTok{] }\OtherTok{=}\NormalTok{ abalone[}\StringTok{\textquotesingle{}rings\textquotesingle{}}\NormalTok{] }\SpecialCharTok{+} \FloatTok{1.5}
\end{Highlighting}
\end{Shaded}

\hypertarget{question-2}{%
\subsubsection{Question 2}\label{question-2}}

Split the abalone data into a training set and a testing set. Use
stratified sampling. You should decide on appropriate percentages for
splitting the data.

\begin{Shaded}
\begin{Highlighting}[]
\FunctionTok{set.seed}\NormalTok{(}\DecValTok{91362}\NormalTok{)}
\NormalTok{abalone\_split }\OtherTok{\textless{}{-}} \FunctionTok{initial\_split}\NormalTok{(abalone, }\AttributeTok{prop =} \FloatTok{0.80}\NormalTok{, }\AttributeTok{strata =} \StringTok{\textquotesingle{}age\textquotesingle{}}\NormalTok{)}
\NormalTok{abalone\_train }\OtherTok{\textless{}{-}} \FunctionTok{training}\NormalTok{(abalone\_split)}
\NormalTok{abalone\_test }\OtherTok{\textless{}{-}} \FunctionTok{testing}\NormalTok{(abalone\_split)}
\end{Highlighting}
\end{Shaded}

\hypertarget{question-3}{%
\subsubsection{Question 3}\label{question-3}}

Using the \textbf{training} data, create a recipe predicting the outcome
variable, \texttt{age}, with all other predictor variables. Note that
you should not include \texttt{rings} to predict \texttt{age}. Explain
why you shouldn't use \texttt{rings} to predict \texttt{age}.

Steps for your recipe:

\begin{enumerate}
\def\labelenumi{\arabic{enumi}.}
\item
  dummy code any categorical predictors
\item
  create interactions between

  \begin{itemize}
  \tightlist
  \item
    \texttt{type} and \texttt{shucked\_weight},
  \item
    \texttt{longest\_shell} and \texttt{diameter},
  \item
    \texttt{shucked\_weight} and \texttt{shell\_weight}
  \end{itemize}
\item
  center all predictors, and
\item
  scale all predictors.
\end{enumerate}

You'll need to investigate the \texttt{tidymodels} documentation to find
the appropriate step functions to use.

\begin{Shaded}
\begin{Highlighting}[]
\NormalTok{simple\_abalone\_recipe }\OtherTok{\textless{}{-}} \FunctionTok{recipe}\NormalTok{(age }\SpecialCharTok{\textasciitilde{}}\NormalTok{ type }\SpecialCharTok{+}\NormalTok{ longest\_shell }\SpecialCharTok{+}\NormalTok{ diameter }\SpecialCharTok{+}\NormalTok{ height }\SpecialCharTok{+}\NormalTok{ whole\_weight }
                                \SpecialCharTok{+}\NormalTok{ shucked\_weight }\SpecialCharTok{+}\NormalTok{ viscera\_weight }\SpecialCharTok{+}\NormalTok{ shell\_weight, }\AttributeTok{data =}\NormalTok{ abalone\_train)}
\NormalTok{abalone\_recipe }\OtherTok{\textless{}{-}} \FunctionTok{recipe}\NormalTok{(age }\SpecialCharTok{\textasciitilde{}}\NormalTok{ type }\SpecialCharTok{+}\NormalTok{ longest\_shell }\SpecialCharTok{+}\NormalTok{ diameter }\SpecialCharTok{+}\NormalTok{ height }\SpecialCharTok{+}\NormalTok{ whole\_weight }\SpecialCharTok{+}\NormalTok{ shucked\_weight}
                         \SpecialCharTok{+}\NormalTok{ viscera\_weight }\SpecialCharTok{+}\NormalTok{ shell\_weight, }\AttributeTok{data =}\NormalTok{ abalone\_train) }\SpecialCharTok{\%\textgreater{}\%} 
                         \FunctionTok{step\_dummy}\NormalTok{(}\FunctionTok{all\_nominal\_predictors}\NormalTok{()) }\SpecialCharTok{\%\textgreater{}\%} 
                         \FunctionTok{step\_interact}\NormalTok{(}\AttributeTok{terms =} \SpecialCharTok{\textasciitilde{}}\NormalTok{ shucked\_weight}\SpecialCharTok{:}\NormalTok{shell\_weight) }\SpecialCharTok{\%\textgreater{}\%} 
                         \FunctionTok{step\_interact}\NormalTok{(}\AttributeTok{terms =} \SpecialCharTok{\textasciitilde{}} \FunctionTok{starts\_with}\NormalTok{(}\StringTok{\textquotesingle{}type\textquotesingle{}}\NormalTok{)}\SpecialCharTok{:}\NormalTok{shucked\_weight) }\SpecialCharTok{\%\textgreater{}\%}
                         \FunctionTok{step\_interact}\NormalTok{(}\AttributeTok{terms =} \SpecialCharTok{\textasciitilde{}}\NormalTok{longest\_shell}\SpecialCharTok{:}\NormalTok{diameter)}
\NormalTok{abalone\_recipe }\OtherTok{=}\NormalTok{ abalone\_recipe }\SpecialCharTok{\%\textgreater{}\%} \FunctionTok{step\_normalize}\NormalTok{(}\FunctionTok{all\_predictors}\NormalTok{())}
\end{Highlighting}
\end{Shaded}

Answer: You shouldn't use \texttt{rings} to predict \texttt{age} because
\texttt{age} is already a function of \texttt{rings}.

\hypertarget{question-4}{%
\subsubsection{Question 4}\label{question-4}}

Create and store a linear regression object using the \texttt{"lm"}
engine.

\begin{Shaded}
\begin{Highlighting}[]
\NormalTok{lm\_model }\OtherTok{\textless{}{-}} \FunctionTok{linear\_reg}\NormalTok{() }\SpecialCharTok{\%\textgreater{}\%} \FunctionTok{set\_engine}\NormalTok{(}\StringTok{"lm"}\NormalTok{)}
\end{Highlighting}
\end{Shaded}

\hypertarget{question-5}{%
\subsubsection{Question 5}\label{question-5}}

Now:

\begin{enumerate}
\def\labelenumi{\arabic{enumi}.}
\item
  set up an empty workflow,
\item
  add the model you created in Question 4, and
\item
  add the recipe that you created in Question 3.
\end{enumerate}

\begin{Shaded}
\begin{Highlighting}[]
\NormalTok{lm\_wflow }\OtherTok{\textless{}{-}} \FunctionTok{workflow}\NormalTok{()}
\NormalTok{lm\_wflow }\OtherTok{\textless{}{-}}\NormalTok{ lm\_wflow }\SpecialCharTok{\%\textgreater{}\%} \FunctionTok{add\_model}\NormalTok{(lm\_model)}
\NormalTok{lm\_wflow }\OtherTok{\textless{}{-}}\NormalTok{ lm\_wflow }\SpecialCharTok{\%\textgreater{}\%} \FunctionTok{add\_recipe}\NormalTok{(abalone\_recipe)}
\end{Highlighting}
\end{Shaded}

\hypertarget{question-6}{%
\subsubsection{Question 6}\label{question-6}}

Use your \texttt{fit()} object to predict the age of a hypothetical
female abalone with longest\_shell = 0.50, diameter = 0.10, height =
0.30, whole\_weight = 4, shucked\_weight = 1, viscera\_weight = 2,
shell\_weight = 1.

\begin{Shaded}
\begin{Highlighting}[]
\NormalTok{lm\_fit }\OtherTok{\textless{}{-}} \FunctionTok{fit}\NormalTok{(lm\_wflow, abalone\_train)}
\NormalTok{fem\_abalone }\OtherTok{\textless{}{-}} \FunctionTok{data.frame}\NormalTok{(}\AttributeTok{type =} \StringTok{\textquotesingle{}F\textquotesingle{}}\NormalTok{, }\AttributeTok{longest\_shell =} \FloatTok{0.50}\NormalTok{, }\AttributeTok{diameter =} \FloatTok{0.10}\NormalTok{, }\AttributeTok{height =} \FloatTok{0.30}\NormalTok{, }
                          \AttributeTok{whole\_weight =} \DecValTok{4}\NormalTok{, }\AttributeTok{shucked\_weight =} \DecValTok{1}\NormalTok{, }\AttributeTok{viscera\_weight =} \DecValTok{2}\NormalTok{, }\AttributeTok{shell\_weight =} \DecValTok{1}\NormalTok{)}
\FunctionTok{predict}\NormalTok{(lm\_fit, fem\_abalone)}
\end{Highlighting}
\end{Shaded}

\begin{verbatim}
## # A tibble: 1 x 1
##   .pred
##   <dbl>
## 1  23.8
\end{verbatim}

\hypertarget{question-7}{%
\subsubsection{Question 7}\label{question-7}}

Now you want to assess your model's performance. To do this, use the
\texttt{yardstick} package:

\begin{enumerate}
\def\labelenumi{\arabic{enumi}.}
\tightlist
\item
  Create a metric set that includes \emph{R\textsuperscript{2}}, RMSE
  (root mean squared error), and MAE (mean absolute error).
\item
  Use \texttt{predict()} and \texttt{bind\_cols()} to create a tibble of
  your model's predicted values from the \textbf{training data} along
  with the actual observed ages (these are needed to assess your model's
  performance).
\item
  Finally, apply your metric set to the tibble, report the results, and
  interpret the \emph{R\textsuperscript{2}} value.
\end{enumerate}

\begin{Shaded}
\begin{Highlighting}[]
\NormalTok{abalone\_metrics }\OtherTok{\textless{}{-}} \FunctionTok{metric\_set}\NormalTok{(rmse, rsq, mae)}
\NormalTok{abalone\_train\_res }\OtherTok{\textless{}{-}} \FunctionTok{predict}\NormalTok{(lm\_fit, }\AttributeTok{new\_data =}\NormalTok{ abalone\_train }\SpecialCharTok{\%\textgreater{}\%} \FunctionTok{select}\NormalTok{(}\SpecialCharTok{{-}}\NormalTok{age))}
\NormalTok{abalone\_train\_res }\OtherTok{\textless{}{-}} \FunctionTok{bind\_cols}\NormalTok{(abalone\_train\_res, abalone\_train }\SpecialCharTok{\%\textgreater{}\%} \FunctionTok{select}\NormalTok{(age))}
\FunctionTok{head}\NormalTok{(abalone\_train\_res)}
\end{Highlighting}
\end{Shaded}

\begin{verbatim}
## # A tibble: 6 x 2
##   .pred   age
##   <dbl> <dbl>
## 1  9.46   8.5
## 2  7.99   8.5
## 3  9.36   9.5
## 4  9.62   8.5
## 5 10.3    8.5
## 6 10.0    9.5
\end{verbatim}

\begin{Shaded}
\begin{Highlighting}[]
\FunctionTok{abalone\_metrics}\NormalTok{(abalone\_train\_res, }\AttributeTok{truth =}\NormalTok{ age, }\AttributeTok{estimate =}\NormalTok{ .pred)}
\end{Highlighting}
\end{Shaded}

\begin{verbatim}
## # A tibble: 3 x 3
##   .metric .estimator .estimate
##   <chr>   <chr>          <dbl>
## 1 rmse    standard       2.13 
## 2 rsq     standard       0.561
## 3 mae     standard       1.53
\end{verbatim}

\hypertarget{required-for-231-students}{%
\subsubsection{Required for 231
Students}\label{required-for-231-students}}

In lecture, we presented the general bias-variance tradeoff, which takes
the form:

\[
E[(y_0 - \hat{f}(x_0))^2]=Var(\hat{f}(x_0))+[Bias(\hat{f}(x_0))]^2+Var(\epsilon)
\]

where the underlying model \(Y=f(X)+\epsilon\) satisfies the following:

\begin{itemize}
\tightlist
\item
  \(\epsilon\) is a zero-mean random noise term and \(X\) is non-random
  (all randomness in \(Y\) comes from \(\epsilon\));
\item
  \((x_0, y_0)\) represents a test observation, independent of the
  training set, drawn from the same model;
\item
  \(\hat{f}(.)\) is the estimate of \(f\) obtained from the training
  set.
\end{itemize}

\hypertarget{question-8}{%
\subsubsection{Question 8}\label{question-8}}

Which term(s) in the bias-variance tradeoff above represent the
reproducible error? Which term(s) represent the irreducible error?

Answer: The reproducible error terms are \(Var(\hat{f}(x_0))\) and
\([Bias(\hat{f}(x_0))]^2\). The irreducible error is \(Var(\epsilon)\).

\hypertarget{question-9}{%
\subsubsection{Question 9}\label{question-9}}

Using the bias-variance tradeoff above, demonstrate that the expected
test error is always at least as large as the irreducible error.

\$\$ E{[}(y\_0 -
\hat{f}(x\_0))\^{}2{]}=Var(\hat{f}(x\_0))+{[}Bias(\hat{f}(x\_0)){]}\^{}2+Var(\epsilon)\textbackslash{}

=\mathrm{E}\left[\left(\hat{f}\left(\mathbf{x}_{0}\right)-\mathrm{E} \hat{f}\left(\mathbf{x}_{0}\right)\right)^{2}\right]+\left[\mathrm{E}\left[\hat{f}\left(\mathbf{x}_{0}\right)\right]-f\left(\mathbf{x}\_\{0\}\right)\right{]}\^{}\{2\}+\operatorname{Var}(\varepsilon)\textbackslash{}
= 0+0 +Var(\epsilon)\textbackslash{} where: ~~~~~~E{[}(y\_0 -
\hat{f}(x\_0))\^{}2{]}~= expected~test~error\textbackslash{}
\{Var\}(\varepsilon) = irreducible~error \$\$

\hypertarget{question-10}{%
\subsubsection{Question 10}\label{question-10}}

Prove the bias-variance tradeoff.

Hints:

\begin{itemize}
\tightlist
\item
  use the definition of \(Bias(\hat{f}(x_0))=E[\hat{f}(x_0)]-f(x_0)\);
\item
  reorganize terms in the expected test error by adding and subtracting
  \(E[\hat{f}(x_0)]\)
\end{itemize}

\[
E[(y_0 - \hat{f}(x_0))^2]=Var(\hat{f}(x_0))+[Bias(\hat{f}(x_0))]^2+Var(\epsilon)\\
=\mathrm{E}\left[\left(\hat{f}\left(\mathbf{x}_{0}\right)-\mathrm{E} \hat{f}\left(\mathbf{x}_{0}\right)\right)^{2}\right]+\left[\mathrm{E}\left[\hat{f}\left(\mathbf{x}_{0}\right)\right]-f\left(\mathbf{x}_{0}\right)\right]^{2}+\operatorname{Var}(\varepsilon)\\ = 0 +Var(\epsilon)\\ {\Rightarrow}\ Var(\hat{f}(x_0))+[Bias(\hat{f}(x_0))]^2 =\ 0\\{\Rightarrow}\ Var(\hat{f}(x_0))=-[Bias(\hat{f}(x_0))]^2\\\ \\therefore,\\ [Bias(\hat{f}(x_0))]^2\ {\Uparrow}\ when\ Var(\hat{f}(x_0))\ {\Downarrow}\\ Var(\hat{f}(x_0))\ {\Uparrow}\ when\ [Bias(\hat{f}(x_0))]^2\ {\Downarrow}
\]

\end{document}
